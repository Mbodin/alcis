\documentclass{article}

\usepackage[utf8]{inputenc}
\usepackage[T1]{fontenc}
\usepackage[english]{babel}
\usepackage{a4wide}
\usepackage{pstricks}
\usepackage{lmodern}
\usepackage{stmaryrd}
\usepackage{amsmath, amsthm}
\usepackage{amsfonts, amssymb}
\usepackage{graphicx}
\usepackage{multirow}
\usepackage{array}
\usepackage{hyperref}
\usepackage{wasysym}

\definecolor{darkblue}{rgb}{0,0,.5}
\hypersetup{unicode=true, colorlinks=false}

\renewcommand{\ge}{\geqslant}
\renewcommand{\le}{\leqslant}
\renewcommand{\preceq}{\preccurlyeq}
\renewcommand{\succeq}{\succcurlyeq}

\newcommand{\Alcis}{\textsc{Alcis}~}
\renewcommand{\C}{\textsc{C}~}

\newcommand{\FIXME}{~\textbf{FIXME}~}

\title{\Alcis Standards}
\author{Martin \textsc{Bodin}}
\date{}


\begin{document}

\maketitle

\newpage

\tableofcontents

\newpage

\paragraph{Introduction}
{
    The language \Alcis is compiled to a \C code.
    It’s main feature is to permit the programmer to create notations at will.
    It’s a strong typed, object-oriented language.
    It has been design to be soft and easy to code with.
}

\section{Files Types}

Here are the list of all the type of \Alcis files there exists:
\begin{itemize}
    \item \Alcis Header. It contains prototypes of functions and variables, and contains parsing informations.
    \item \Alcis Source Code. It contains declarations of functions and variables.
    \item \Alcis Compiled Interface. It contains the \C equivalent of a group of function and variables.
\end{itemize}

There is no extensions forced with thoses different types of files\footnote
{
    In fact \Alcis is design to manipulate extensionless files.
}, but in the case an extension is needed (for instance in the case of the use of a Makefile), here is the advised extension for the \Alcis files:

\begin{center}\begin{tabular}{|c||c|}
\hline File & Extension \\
\hline \Alcis Header & \tt .ah \\
\hline \Alcis Source Code & \tt .ac \\
\hline \Alcis Compiled Interface & \tt .ai \\
\hline \C Header & \tt .h \\
\hline \C Source Code & \tt .c \\
\hline
\end{tabular}\end{center}

\section{\Alcis Header}

\FIXME

\section{\Alcis Source Code}

\FIXME

\section{\Alcis Compiled Interface}

\FIXME

\end{document}

